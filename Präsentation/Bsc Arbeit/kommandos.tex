\usepackage{multirow}
\usepackage{listings}
\usepackage{color}

%
% Kommandos für das Zitieren.
%
\newcommand{\zitiereSeite}[2]{\cite[][S.#2]{#1}}
\newcommand{\zitiereKapitel}[2]{\cite[][Kap.#2]{#1}}
\newcommand{\zitiere}[1]{\cite[]{#1}}

\newcommand{\vergleiche}[2]{(nach \cite[][S.#2]{#1})}

\newcommand{\sieheSeite}[2]{(siehe \cite[][S.#2]{#1})}
\newcommand{\sieheKapitel}[2]{(siehe \cite[][Kap.#2]{#1})}
\newcommand{\siehe}[1]{(siehe \cite{#1})}
%
% Kurzkommandos für Abbildungen, sowohl zum Referenzieren als auch zum Einfügen
%

%
% Kommandos für Textersetzungen, Fett, Kursiv, Mehrzeilig,...
%
\newcommand{\dbcommand}[1]{\textit{#1}}
\newcommand{\dbfile}[1]{\textit{#1}}
\newcommand{\dbname}[1]{\textbf{#1}}

\newcommand{\klasse}[1]{\textbf{#1}}
\newcommand{\member}[1]{\textit{#1}}
\newcommand{\funktion}[1]{\texttt{#1}}

\newcommand{\controller}[1]{\textbf{#1}}
\newcommand{\resource}[1]{\textit{#1}}


\definecolor{dkgreen}{rgb}{0,0.6,0}
\definecolor{gray}{rgb}{0.5,0.5,0.5}
\definecolor{mauve}{rgb}{0.58,0,0.82}

\lstset{frame=tb,
  language=Java,
  aboveskip=3mm,
  belowskip=3mm,
  basicstyle={\small\ttfamily},
  numberstyle=\tiny\color{gray},
  keywordstyle=\color{blue},
  commentstyle=\color{dkgreen},
  stringstyle=\color{mauve},
  captionpos=b
}
\renewcommand{\lstlistingname}{Ausschnitt}
\renewcommand{\lstlistlistingname}{List von \lstlistingname en}