\chapter[Kurzfassung] {Kurzfassung}       
\label{cha:kurzfassung}

In dieser Arbeit werden die Möglichkeiten zur parallelen Entwicklung in Android untersucht. Dazu wird zuerst ein Überblick gegeben, was Android ist und wie es entstanden ist. Danach wird die parallele Programmierung selbst erklärt. Besonders Wert wird hierbei auf potenzielle Fehlerquellen gelegt, welche nur bei parallelen Applikationen auftreten. Im Anschluss wird die verwendete Entwicklungsumgebung näher beschrieben. Hierbei werden hauptsächlich Funktionen betrachtet, die im späteren Verlauf der Arbeit verwendet werden. Die letzten beiden Kapitel beschäftigen sich mit den Ergebnissen der praktischen Programmieraspekte der Arbeit. Als erstes werden die Möglichkeiten einen Deadlock in einer Applikation zu finden näher beschrieben. Der zweite praktische Teil dieser Arbeit beschäftigt sich mit den verschiedenen Möglichkeiten parallele Software in Android zu entwickeln. In diesem Kapitel werden die verwendeten Methoden nach Performance verglichen.
\\
\\
Das Ziel dieser Arbeit ist es einen Überblick über parallele Software in Android zu geben und auf potenzielle Gefahren, beziehungsweise Fehlerquellen hinzuweisen. Zusätzlich dazu sollen auch etwaige Lösungsansätze aufgezeigt werden. Im Abschluss soll noch ein kurzer Überblick, in Form eines Performancevergleichs, über die verschiedenen Möglichkeiten zur parallelen Entwicklung in Android gegeben werden.