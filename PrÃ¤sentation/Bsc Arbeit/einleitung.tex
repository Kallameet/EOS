\chapter[Einleitung] {Einleitung}       
\label{cha:einleitung}

Erweiternd zur Lehrveranstaltung Parallele Software / Systemprogrammierung, im 4. Semester des Fachhochschulstudiengangs Hardware Software Design, wird in dieser Arbeit das mobile Betriebssystem Android näher untersucht. Besonderer Wert wird hierbei auf die verschiedenen Möglichkeiten der parallelen Programmierung in Android gelegt.
\\
\\
Der erste Teil der Arbeit beschäftigt sich mit den theoretischen Grundlagen und soll einen allgemeinen Überblick über Android, den Grundbegriffen der parallelen Programmierung und der Entwicklungsumgebung Android Studio verschaffen. Im zweiten Teil der Arbeit wird der praktische Aspekt behandelt. Hierbei werden zuerst die Möglichkeiten einen Deadlock in einer Applikation zu finden näher beschrieben. Der zweite Teil beschäftigt sich mit den verschiedenen Möglichkeiten der Parallelisierung von Software in Android. Diese werden im Bezug auf Performance miteinander verglichen.
\\
\\
Das Ziel dieser Arbeit ist es einen allgemeinen Überblick über Android und den Möglichkeiten parallele Software dafür zu entwickeln zu geben. Des Weiteren soll auf potenzielle Gefahren, beziehungsweise Fehlerquellen von paralleler Software hingewiesen werden. Zusätzlich dazu sollen auch etwaige Lösungsansätze, mithilfe der von Android Studio zur Verfügung gestellten Mitteln, aufgezeigt werden. Im Abschluss soll noch ein kurzer Überblick, in Form eines Performancevergleichs, über die verschiedenen Möglichkeiten zur parallelen Entwicklung in Android gegeben werden.
\\
\\
Die Auswahl fiel auf dieses Thema, da ich ein großes Interesse an Android habe und mich bereits vor Beginn der Arbeit näher damit auseinander setzen wollte. Mit dieser Arbeit bot sich somit mit eine gute Möglichkeit, das mobile Betriebssystem näher kennenzulernen, an. Ein weiterer Punkt der die Auswahl dieses Themas bestärkte, war das Interesse daran, zu sehen, wie gut parallele Software auf mobilen Geräten bereits unterstützt wird.